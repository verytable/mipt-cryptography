\documentclass[a4paper]{article}

\usepackage{cmap}                   % поиск в PDF
\usepackage[T2A]{fontenc}           % кодировка
\usepackage[english,russian]{babel} % локализация и переносы
\usepackage[utf8x]{inputenc}
\usepackage{amsmath,amsthm}
\usepackage{multirow}

\title{Криптография, Лекция № 12}

\theoremstyle{definition}
\newtheorem{definition}{Definition}

\theoremstyle{plain}
\newtheorem{theorem}{Theorem}
\newtheorem{lemma}{Lemma}
\newtheorem{remark}{Remark}
\newtheorem{claim}{Claim}
\newtheorem{idea}{Idea}     
\newtheorem{example}{Example} 
      
\begin{document}
\maketitle

\section{Задача Безопасных Вычислений}

Есть некоторое количество участников $A_i$ и каждый из них что-то знает $x_i$,
и что-то хочет узнать $f_i(x_1, \ldots, x_n)$. Задача: все стороны должны узнать
$f_i(x_1, \ldots, x_n)$ но ничего сверх того.

\begin{example}~\\
	$A$ имеет $x_1$ рублей,
	$B$ имеет $x_2$ рублей;
	и они хотят узнать у кого больше $f_1(x_1, x_2) = f_2(x_1, x_2) = I(x_1 \ge x_2)$
\end{example}

\begin{definition}~\\
	Протокол решения задачи - набор из $n$ полиномиальных вероятностных
	интерактивных алгоритмов.
\end{definition}

\noindent В идеальной модели есть независимая доверенная сторона, все ей отправляют $x_i$,
она возвращает $f_i(x_1, \ldots, x_n)$.~\\

\noindent Парадигма: в реальной модели не должно быть возможным ничего, что
невозможно в идеальной.~\\

\noindent Есть несколько видов поведения агентов:
\begin{itemize}
	\item Честное поведение:~\\
		Даже если агент узнал что-то не то, он сразу же "забывает"
	\item Полу-честное (Semi-honest):~\\
		В идеальной модели агент может что-то вычислить на основе входа и выхода.
		В реальной модели агент действует по протоколу, но на базе промежуточной,
		входной и выходной информации может что-то вычислить.
	\item Нечестное (Malicious):~\\
		В идеальной модели агент может подменить ход, или вообще отказаться участвовать.
		В реальной модели агент может делать что угодно (действует не про протоколу,
		подменяет вход), только в условиях полиномиальной ограниченности.
\end{itemize}


\begin{definition} (Надежность в полу-честной модели)~\\
	$\Pi$ - протокол, $VIEW_i^{\Pi}(x, y)$ - все сообщения, которые получает сторона $i$
	при выполнении протокола $\Pi$ на входе $(x, y)$.~\\
	$\Pi$ надёжен в полу-честной модели, если при любом $i \ in \{1, 2\}$ существует
	полиномиальный алгоритм $S_i$ такой, что
	$$
		S_i(x_i, f_i(x1, x_2)) \textup{вычислительно не отличима от} VIEW_i^{\Pi}(x, y)
	$$
\end{definition}

\begin{remark}~\\
	Аналогично записывается определение надежности для более чем двух сторон.
\end{remark}

\begin{definition} (Слепая предача (oblivious transfer), $OT_1^k$)~\\
	Вход: $x = (\sigma_1, \ldots, \sigma_k), \sigma_j \in \{0, 1\}$ $y = i \in \{1, \ldots, k\}$.~\\
	Выход: $f_1(x, y) = f(x, y) = \varepsilon$; $f_2(x, y) = g(x, y) = \sigma_i$
\end{definition}

\begin{example} (Применение $OT_1^4$)~\\
	Пусть $f$ вычисляется арифметической схемой, работающей с битами, то есть
	схемой из функциональных элементов $\bigwedge$ и $\oplus$. Агент $A$ выбирает
	случайное $r$, запоминает его и посылает $r \odot x$.
	После этого $A$ имеет $r$, $B$ имеет $r'$; $r, r'$ - случайные, но $r \oplus r' = x$.
	Далее $B$ выбирает случайное $s$, запоминает его и передает $s \oplus y$.~\\
	
	\noindent Сложение:~\\
		$A$ знает $a_1, b_1$, $a_1 \oplus a_2 = a$~\\
		$B$ знает $a_2, b_2$, $b_1 \oplus b_2 = b$~\\
		Пусть $c_1 = (a_1 \oplus b_1)$, $c_2 = (a_2 \oplus b_2)$, тогда
		$c_1 \oplus c_2 = (a_1 \oplus b_1) \oplus (a_2 \oplus b_2) = a \oplus b$~\\
		
	\noindent Умножение:~\\
		$A$ знает $a_1, b_1$, $a_1 \oplus a_2 = a$~\\
		$B$ знает $a_2, b_2$, $b_1 \oplus b_2 = b$~\\
		Нужно:~\\
			$A$ получил $c_1$
			$B$ получил $c_2$ такие, что $c_1 \oplus c_2 = (a_1 \oplus b_1) \bigwedge (a_2 \oplus b_2)$~\\
			
		\noindent $A$ выбирает $c_1$ случайно~\\
		Вычисляет $\sigma_1 = c_1 \oplus a_1 \cdot b_1$ - для $a_2 = 0, b_2 = 0$~\\
		$\sigma_2 = c_1 \oplus a_1 \cdot (b_1 \oplus 1)$ - для $a_2 = 0, b_2 = 1$ ~\\
		$\sigma_3 = c_1 \oplus (a_1 \oplus 1) \cdot b_1$ - для $a_2 = 1, b_2 = 1$ ~\\
		$\sigma_4 = c_1 \oplus (a_1 \oplus 1) \cdot (b_1 \oplus 1)$ - для $a_2 = 1, b_2 = 1$ ~\\
		Итого $A$ знает $4$ варианта, $B$ знает, какой из них нужен, вот и получается задача $OT_1^4$.
\end{example}

\subsection{Протокол $OT_1^k$}

$S$ выбирает $(\alpha, \tau)$ и посылает $\alpha$. Где $\alpha$ - номер одностороней
перестановки, $\tau$ - система для ее обращения.~\\
$R$ выбирает $x_1, \ldots, x_k \in D_{\alpha}$ - область определения перестановки $f_{\alpha}$.
И при $i \ne j \rightarrow y_j = x_j; i = j \rightarrow y_j = f_{\alpha}(x_i)$.
Посылает $y_1, \ldots, y_k$.~\\
$S$ вычисляет $f_{\alpha}^{-1}(y_j)$, $\tau_j = \sigma_j \oplus b_j(z_j)$, где
$b_{\alpha}$ - трудный бит.
Отправляет $\tau_1, \ldots, \tau_k$.~\\
Ответ: $\tau_i \oplus b(x_i)$







































\end{document}
