\documentclass[a4paper]{article}

\usepackage{cmap}                   % поиск в PDF
\usepackage[T2A]{fontenc}           % кодировка
\usepackage[english,russian]{babel} % локализация и переносы
\usepackage[utf8x]{inputenc}
\usepackage{amsmath,amsthm}
\usepackage{multirow}

\title{Криптография, Лекция № 5}

\theoremstyle{definition}
\newtheorem{definition}{Definition}

\theoremstyle{plain}
\newtheorem{theorem}{Theorem}
\newtheorem{lemma}{Lemma}
\newtheorem{remark}{Remark}
\newtheorem{claim}{Claim}
\newtheorem{idea}{Idea}     
\newtheorem{example}{Example} 
      
\begin{document}
\maketitle

\noindent Доведем рассуждения с прошлой лекции. В общих чертах: генератор случайных числе - функция,
котора берет короткое число независимых случайных битов и выдает длинную строку, чтобы отличить
которую нужно затратить много ресурсов.~\\

\begin{theorem}(Голдрайх-Левин)~\\
	Если $f: \{0, 1\}^{n} \rightarrow \{0, 1\}^n$ - односторонняя перестановка, то
	$g: \{0, 1\}^{2n} \rightarrow \{0, 1\}^{2n}$, $g(x, y) = (f(x), y)$ - тоже односторонняя перестановка,
	а $h(x, y) = x \odot y$ - трудный бит для $g$.
\end{theorem}

\begin{theorem}(О кодах Адамара или о списочном декодировании)~\\
	Пусть $h_x(y) = x \odot y$ - код Адамара слова $x \in \{0, 1\}^n$.
	Пусть $\bar{h}(y)$ - такая функция, что доля $y$ удовлетоворяющих $\bar{h}(y) = h_x(y)$ больше либо равна $\frac{1}{2}$.
	Тогда существует вероятностный алгоритм, работающий $poly(\frac{n}{\varepsilon})$, имеющий
	произвольный доступ к $\bar{h}$, выдающий список слов длины $n$, с вероятностью $\ge \frac{1}{2}$
	содержит $x$.
\end{theorem}

\noindent Почему это называется декодированием списка? Вообще кодирование нужно для исправления ошибок. Код
Адамара - длинный код. Точного полиномиального алгоритма декодирования не существует.
$h$ - код, $\bar{h}$ - испорченный код.~\\

\begin{theorem}~\\
	Из теоремы о списочном декодировании выводится теорема Голдрайха-Левина.
\end{theorem}

\begin{proof}~\\
	Пусть существует схема, которая с вероятностью $\ge \frac{1}{2} + \varepsilon$ по $(f(x), y)$
	находит $x \odot y$.
	$$
		Pr_y\{C(f(x), y) = x \odot y \} \ge \frac{1}{2} + \varepsilon
	$$
	$$
		Pr_y\{\bar{C}(y) = h_x(y) \} \ge \frac{1}{2} + \varepsilon
	$$
	$\bar{C}$ - искаженный код Адамара слова $x$.~\\
	
	
	\noindent Обратитель $f$:
	\begin{enumerate}
		\item Применяет алгоритм списочного декодирования для $\bar{h} = \bar{C}$ и $\varepsilon$
		\item Вычисляет значения $f$ на всех элементах полученного списка
		\item Если $f(\bar{x}) = f(x)$, то выводит $\bar{x}$, иначе выводит что угодно.
	\end{enumerate}
	Утверждается, что этот обратитель будет достаточно успешным. Он точно будет работать
	полиномиальное время (он обращается к схеме, нужно переделать алгоритм в полиномиальную схему).
	Нужно понять, с какой вероятностью он будет обращать.
	Эта вероятность будет равна $\frac{\varepsilon}{4}$. Ибо если декодирование успешно, то
	нужный $x$ будет найден и вероятность успеха декодирования будет $\frac{1}{2}$.
	Нужно посчитать долю $x$, для которых
	$$
		Pr_y\{\bar{C}(y) = h_x(y)\} \ge \frac{1}{2} + \frac{\varepsilon}{2}	
	$$
	Эта доля $\ge \frac{\varepsilon}{2}$, иначе
	$$
		Pr_y\{C(f(x), y) = h_x(y)\} < \frac{\varepsilon}{2}\cdot 1 + 1\cdot(\frac{1}{2} + \frac{\varepsilon}{2}) = \frac{1}{2} + \varepsilon	
	$$
	Получили противоречие с предположением.~\\
	
	\noindent $\ge \frac{\varepsilon}{2}$ - вероятность хорошего $x$;~\\
	
	\noindent $\ge \frac{1}{2}$ - вероятность успешного обращения хорошего $x$~\\
	
	\noindent Из последних двух оценок поулчаем $\ge \frac{\varepsilon}{4}$ - вероятность
	успешного обращения $x$.
\end{proof}

\begin{lemma}(ЗБЧ для попарно независимых случайных величин)~\\
	Пусть $\xi_1, \ldots, \xi_N$ - попарно независимые одинакого распределенные,
	бернулиевские случайные величины.
	$E\xi_i = \frac{1}{2} - \varepsilon$. Тогда
	$$
		Pr\{\xi_1 + \ldots \xi_N \ge \frac{1}{2}N\} \le \frac{1}{\varepsilon^2}\frac{1}{N}	
	$$
\end{lemma}

\begin{proof}~\\
	Следствие попарной независимости:
	$$
		D(\xi_1 + \ldots + \xi_N) = D\xi_1 + \ldots + D\xi_N	
	$$
	$$
		E(\xi_1 + \ldots + \xi_N) =(\frac{1}{2} - \varepsilon)N	
	$$
	Применим неравенство Чебышёва:
	$$
		Pr\{\xi_1 + \ldots + \xi_N \ge E(\xi_1 + \ldots + \xi_N) + \varepsilon N\} \le \frac{D(\xi_1 + \ldots + \xi_N)}{(\varepsilon N)^2} \le \frac{N \cdot \frac{1}{4}}{(\varepsilon N)^2} =	\frac{1}{4N\varepsilon^2}
	$$
\end{proof}

\begin{proof}(теоремы о списочном декодировании)~\\
	$$
		h_x(y) = h_x(y + r) + h_x(r)	
	$$
	Отсюда еще можно заметить, что если бы было испорчено не более четверти битов, то
	можно было $h_x(y), h_x(r)$ заменить на $\bar{h}$ и выбирать максимум.
	$$
		h_x(y) = majority_r(\bar{h}(y + r) + h_x(r))	
	$$
	Можно взять выборку из попарно независимых $r_i$. Как их построить?
	Есть небольшое количество случайных вектор-строк длины $n$: $u_1, \ldots, u_s$.~\\
	$r_1, \ldots, r_{2^s - 1}$ - все нетривиальные суммы $u_j$. Они будут
	попарно независимыми равномерно распределенными.
	Представим $r_i$ в виде вектор-столбца $R$. $R = A \cdot U$.~\\
	
	\noindent $h_x(u_1 + u_2 + u_s) = h_x(u_1) + h_x(u_2) + h_x(u_s)$.
	То есть $h_x(r_i)$ полностью определяется $h_x(u_1), \ldots, h_x(u_s)$.~\\
	
	\noindent $k$-ый бит $x$: $x_k = h_k(e_k) = x \odot e_k$~\\	
	
	\noindent Алгоритм восстановления $x$ (списком):
	\begin{enumerate}
		\item Для всех строк длины $b \in \{0, 1\}^s$
		\item Для всех $k = 1, \ldots, n$~\\
			$$
				x_k = majority\vert_{i = 1, \ldots, 2^s - 1}(\bar{h}(e_k + r_i) + h_x(r_i))		
			$$
			где $h_x(r_i)$ посчитан при условии $h_x(u_1) = b_1, \ldots, h_x(u_s) = b_s$
		\item Добавить $x$ в список.
	\end{enumerate}
	
	\noindent Введем случайные перменные $\xi_i$ - индикатор $\bar{h}(y + r_i) = h_x(y + r_i)$.
	$r_i$ - случайные, следовательно $Pr\{\xi_i = 1\} \ge \frac{1}{2} + \varepsilon$.
	Посчитаем вероятность плохого исхода: $\le \frac{1}{4\varepsilon^2}\cdot\frac{1}{2^s - 1}\cdot n$.
	Хотим, чтобы последняя величина была меньше $\frac{1}{2}$. Откуда $s \simeq \log \frac{n}{\varepsilon^2}$. Весь алгоритм, как легко проверить, полиномиальный.
\end{proof}

\end{document}
