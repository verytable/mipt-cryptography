\documentclass[a4paper]{article}

\usepackage{cmap}                   % поиск в PDF
\usepackage[T2A]{fontenc}           % кодировка
\usepackage[english,russian]{babel} % локализация и переносы
\usepackage[utf8x]{inputenc}
\usepackage{amsmath,amsthm}
\usepackage{multirow}

\title{Криптография, Лекция № 3}

\theoremstyle{definition}
\newtheorem{definition}{Definition}

\theoremstyle{plain}
\newtheorem{theorem}{Theorem}
\newtheorem{lemma}{Lemma}
\newtheorem{remark}{Remark}
\newtheorem{claim}{Claim}
\newtheorem{idea}{Idea}      
\newtheorem{example}{Example} 
      
\begin{document}
\maketitle

\section{Генераторы псевдо-случайных чисел}

\begin{definition}~\\
	$\alpha_n$, $\beta_n$ вычислительно не отличимые семейства случайных величин, если
	$$
		\forall p(\cdot)\ \forall \{D_n\}\ \exists N\ \forall n > N\ |Pr\{D_n(\alpha_n) = 1\} - Pr\{D_n(\beta_n) = 1\}| < \frac{1}{p(n)}	
	$$
	где $p(\cdot)$ - полином, $D_n$ - схема полиномиального размера.
\end{definition}

\begin{lemma}~\\
	\begin{enumerate}
		\item Вычислительная неотличимость - отношение эквивалентности. (Сумма двух функций, которые стремятся к нулю быстрее любого полинома обладет тем же свойством).
		\item $\alpha_n \sim \beta_n \Rightarrow \alpha_n \gamma_m \sim \beta_n \gamma_n$, $\gamma_n$ не зависит от $\alpha_n$ и $\beta_n$~\\
		\begin{eqnarray*}
			&& |Pr_{\alpha, \gamma}\{D_n(\alpha_n \gamma_n) = 1\} - Pr_{\beta, \gamma}\{D_n(\beta_n \gamma_n)| =      \nonumber \\
   			& = & |E_{\gamma}(Pr_{\alpha}\{D_n(\alpha_n \gamma_n) = 1\} - Pr_{\beta}\{D_n(\beta_n \gamma_n) = 1\})| > \frac{1}{q(n)} \nonumber \\
        \end{eqnarray*}
		\item $\alpha_n \sim \beta_n \Rightarrow C_n(\alpha_n) \sim C_n(\beta_n)$, $C_n$ - схема полиномиального размера.~\\
		
		Если $D_n$ отличает $C_n(\alpha)$ от $C_n(\beta)$, то $D_n \circ C_n$ отличает $\alpha_n$ от $\beta_n$
	\end{enumerate}
\end{lemma}

\begin{definition}~\\
	$G_n \colon \{0, 1\}^{k(n)} \rightarrow \{0, 1\}^{l(n)}$ - \textbf{генератор псевдо-случайных чисел},
	если случайные величины $G_n(U_{k(n)})$ и $G_n(U_{l(n)})$ вычислительно неотличимы то есть:
	$$
		\forall p(\cdot)\ \forall \{D_n\}\ \exists N\ \forall n > N\ |Pr\{D_n(\alpha_n) = 1\} - Pr\{D_n(\beta_n) = 1\}| < \frac{1}{p(n)}	
	$$
\end{definition}

\begin{remark}~\\
	\begin{itemize}
		\item Важна именно вычислительная неотличимость. Заменить ее на статистическую неотличимость не получится.
		\item $G_n$ - сильно одностороння функция.~\\	
		Пусть $\ldots Pr\{G_n(R_n(G_n(x))) = G_n(x)\} > \varepsilon$.~\\
		$D_n(y) = if(G_n(R_n(y)) = y)\ 1, else\ 0$
	\end{itemize}
\end{remark}

\begin{definition}~\\
	$\beta(x)$ - \textbf{Трудный бит} для $f$, если
	$$
		\forall p\ \forall \{C_n\}\ \exists N\ \forall n > N\ |Pr\{C_n(f(x)) = \beta(x)\} - \frac{1}{2}| < \frac{1}{p(n)}	
	$$
\end{definition}

\begin{example}(Предположительный)~\\
	Функция Рабина: $(x, y) \mapsto (x^2 mod y, y)$. Трудный бит - четность $x$. 
\end{example}

\noindent План доказательста существования генератора случайных битов, если известно, что существует одностороння функция:
\begin{itemize}
	\item Односторонняя перестановка $\mapsto$ одностороння перестановка с трудным битом.~\\
	$g(x, y) = (f(x), y)$~\\
	$\beta(x, y) = x \odot y = \oplus_{i = 1}^{n} x_i y_i$
	\item Односторонняя перестановка с трудным битом $\mapsto$ генератор
	\item Генератор($n \rightarrow n + 1$) $\mapsto$ Генератор($n \rightarrow p(n)$)
\end{itemize}

\begin{lemma}~\\
	$\beta(x)$ - трудный бит для $f(x) \Leftrightarrow x \mapsto f(x)\beta(x)$ - генератор.
\end{lemma}

\begin{proof}~\\
	$\Leftarrow$~\\
	$\beta$ можно угадать $\Rightarrow$ $f(x)\beta(x)$ можно отличить от $y$.~\\
	$D_n(y) = (C_n(y \mid_{1 \ldots n}) \leftrightarrow y_{n + 1})$~\\
	$y$ случайное $\Rightarrow$ $D_n$ угадывает с вероятностью $\frac{1}{2}$.~\\
	$y = f(x)\beta(x) \Rightarrow D_n$ угадает с вероятностью $\frac{1}{2} + \frac{1}{p(n)}$~\\
	$\Rightarrow$~\\
	$f(x)\beta(x)$ умеем отличать от $y$ $\Rightarrow$ можно предсказать $\beta(x)$ по $f(x)$.~\\
	$y$ можно представить в виде $f(x)r$.
	
	
	\begin{tabular}{cc|c|c|c|c|l}
		\cline{3-4}
		& & \multicolumn{2}{ c| }{$D(f(x)1)$} \\ \cline{3-4}
		& & 0 & 1 \\ \cline{1-4}
		\multicolumn{1}{ |c  }{\multirow{2}{*}{$D(f(x)0)$} } &
		\multicolumn{1}{ |c| }{0} & random & 1 \\ \cline{2-4}
		\multicolumn{1}{ |c  }{}                        &
		\multicolumn{1}{ |c| }{1} & 0 & random \\ \cline{1-4}
	\end{tabular}~\\
	
	\noindent Введем $p_0, p_1, q_0, q_1, r_0, r_1, s_0, s_1$ - вероятности
	в соответствующих условиях того, что $\beta(x) = 0$ и $\beta(x) = 1$~\\
	Например: $r_1 = Pr\{\beta(x) = 1,\ D(f(x)0) = 1,\ D(f(x)1) = 0\}$.~\\	
	
	$Pr\{C_n(f(x)) = \beta(x)\} = \frac{1}{2}(p_0 + p_1) + q_1 + r_0 + \frac{1}{2}(s_0 + s_1)$ \textcircled{=}~\\
	
	$Pr\{D(f(x)r) = 1\} = (s_0 + s_1) + \frac{1}{2}(r_0 + r_1) + \frac{1}{2}(q_0 + q_1)$~\\
	
	$Pr\{D(f(x)\beta(x)) = 1\} = (s_0 + s_1) + r_0 + q_1$~\\
	
	$\frac{1}{2}(r_0 - r_1) + \frac{1}{2}(q_1 - q_0) > \varepsilon$~\\
	
	\begin{eqnarray*}
		& \textcircled{=} & \frac{1}{2}(p_0 + p_1) + \frac{1}{2}(q_1 + q_0) + \frac{1}{2}(r_0 + r_1) + \frac{1}{2}(r_0 - r_1) + \frac{1}{2}(s_0 + s_1) = \nonumber \\
		& = & \frac{1}{2} + \frac{1}{2}(q_1 - q_0) + \frac{1}{2}(r_0 - r_1) > \frac{1}{2} + \varepsilon
	\end{eqnarray*}		
	
\end{proof}

\begin{idea} Последнего шага~\\
	$x \mapsto g(x)h(x) \mapsto g(g(x))h(g(x))h(x) \mapsto g(g(g(x)))h(g(g(x)))h(g(x))h(x) \mapsto \ldots$
\end{idea}

\end{document}
