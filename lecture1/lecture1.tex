\documentclass[a4paper]{article}

\usepackage{cmap}                   % поиск в PDF
\usepackage[T2A]{fontenc}           % кодировка
\usepackage[english,russian]{babel} % локализация и переносы
\usepackage[utf8x]{inputenc}
\usepackage{amsmath,amsthm}

\title{Криптография, Лекция № 1}

\theoremstyle{definition}
\newtheorem{definition}{Definition}

\theoremstyle{plain}
\newtheorem{theorem}{Theorem}
\newtheorem{lemma}{Lemma}
\newtheorem{remark}{Remark}
\newtheorem{claim}{Claim}
\newtheorem{idea}{Idea}      
      
\begin{document}
\maketitle

\begin{definition}~\\
    Функция $f\colon \{0, 1\}^{k(n)} \rightarrow \{0, 1\}^{l(n)}$ называется сильно односторонней, если $f$ вычисляется за время $poly(n)$ и
    $$
        \forall p \forall\{Rn\}_{n = 1}^{\infty} \exists N \forall n > N Pr_{x \in \{0, 1\}^{k(n)}}\{f(Rn(f(x))) = f(x)\} < \frac{1}{p(n)}
    $$
    
  \noindent $Rn$ - семейства схем полиномиального размера
\end{definition}


\begin{definition}~\\
    Функция $f\colon \{0, 1\}^{k(n)} \rightarrow \{0, 1\}^{l(n)}$ называется слабо односторонней, если $f$ вычисляется за время $poly(n)$ и 
    $$
        \exists q \forall\{Rn\}_{n = 1}^{\infty} \exists N \forall n > N Pr_{x \in \{0, 1\}^{k(n)}}\{f(Rn(f(x))) = f(x)\} < 1 - \frac{1}{q(n)}
        $$
    
  \noindent $Rn$ - семейства схем полиномиального размера
\end{definition}


\begin{remark}~\\
    Сильно односторонняя функция является слабо односторонней. Достаточно взять $p = 2$
\end{remark}


\begin{claim}~\\
    Если существует слабо одностороняя функция, то $P \ne NP$
\end{claim}

\begin{proof}~\\
    $L = \{y \in \{0, 1\}^{l(n)} \colon \exists x \in \{0, 1\}^{k(n)} f(x) = y\} \in NP$
    
    \noindent Если $P = NP$, то соответсвующая задача поиска решается за полиномиальное время. А эта задача обращает $f(x)$.
\end{proof}

\begin{claim}~\\
    Если $f$ - одностороння функция, то $h(x) = f(x)0\ldots0$ ($l(n)$ нулей) - односторонняя функция.
\end{claim}

\begin{proof}~\\
    Дан $y$, хотим найти $x\colon f(x) = y$. Запишем $y$ дополненный $l(n)$ нулями, обратим $h$, получим $x$.
\end{proof}

\begin{claim}~\\
    Если $f$ - одностороння функция, то $h(x) = f(x)f(x)$ (конкатенация) - односторонняя функция.
\end{claim}

\begin{proof}~\\
    Дан $y$, хотим найти $x\colon f(x) = y$. Запишем $y$ два раза подряд, обратим $g$, получим $x$.
\end{proof}

\begin{definition}~\\
    Функция $f\colon \{0, 1\}^{k(n)} \rightarrow \{0, 1\}^{l(n)}$ называется сильно односторонней в вероятностной схеме, если $f$ вычисляется за время $poly(n)$ и
    $$
        \forall p \forall\{Rn\}_{n = 1}^{\infty} \exists N \forall n > N Pr_{x \in \{0, 1\}^{k(n)}, r}\{f(Rn(f(x), r)) = f(x)\} < \frac{1}{p(n)}
    $$
    
  \noindent $Rn$ - семейства схем полиномиального размера
\end{definition}

\begin{remark}~\\
    Сильно односторонняя функция является сильно односторонней в вероятностной схеме и наоборот
\end{remark}

Примеры предположительно односторонних функций:
\begin{enumerate}
    \item Умножение (cлабо односторонняя функция)~\\
    $f(x) = x \cdot y$~\\
    $|x| = |y| = n$
    \item SUBSET-SUM~\\
    $(n_1,\ldots,n_k, s) \rightarrow (n_1,\ldots,n_k, N)$, $|s| = k$~\\
    $N = \sum_{i\colon s_i = 1}n_i$
\end{enumerate}

\begin{claim}~\\
    Если существует сильно односторонняя функция, то существует слабо одностороняя функция, не являющаяся сильно односторонней.
\end{claim}

\begin{proof}~\\
    $g(x0) = f(x)0$~\\
    $g(x1) = x1$~\\
    Такая $g$ - слабо односторонняя, ибо если ее можно обратить с вероятностью $\frac{2}{3}$, то ее можно обратить в первом случае с вероятностью $\frac{1}{3}$.
    Тогда и $f$ можно обратить с такой вероятностью, а значит, $f$ - не сильно односторонняя.
\end{proof}

\begin{theorem}~\\
    Если существует слабо одностороняя функция, то существет сильно одностороняя функция.
\end{theorem}

\begin{idea}~\\
    $F(x_1,\ldots,x_N) = f(x_1)\ldots f(x_N)$~\\
    Пусть обратитель $F$ пытается обратить все компоненты по отдельности. Тогда
    вероятность успеха $\le \left(1 - \frac{1}{q(n)}\right)^N = l^{-n}$
\end{idea}

\begin{proof}~\\
    Нужно по рбратиелю $F$ построить обратитель $f$.~\\
    $S(y)\colon$ запуск $R(y, f(x_2),\ldots,f(x_N))$, проверка успешности~\\
    запуск $R(y, f(x_2),\ldots,f(x_N))$, проверка успешности~\\
    запуск $R(y, f(x_2),\ldots,f(x_N))$, проверка успешности~\\
    
    
    \noindent Достаточно доказать утверждение:
    \begin{claim}~\\
        Если $R$ успешен с вероятностью $\ge \frac{1}{p(n)}$, то~\\
        $$
            \forall q \exists m S \textup{успешен с вероятностью} \ge 1 - \frac{1}{q(n)}
        $$
    \end{claim}
    
    $$
        S_1(x) = Pr_{x_2,\ldots,x_n}\{f(R_1(f(x), f(x_2),\ldots,f(x_N))) = f(x)\}
    $$
    
    $$
        S_i(x) = Pr_{x_1,\ldots,x_{i - 1}, x_{i + 1},\ldots,x_N}\{f(R_1(f(x_1),\ldots,f(x_{i - 1}), f(x), f(x_{i + 1}),\ldots,f(x_N))) = f(x)\}
    $$
    Вероятность успеша 1 шага $S \ge max\{S_1(x),\ldots,S_N(x)\}$~\\
    \begin{idea}~\\
        Если $max S_i(x) > \frac{1}{poly(n)}$, тогда за счет выбора $m$ можно повысить вероятность успеха до $1 - \frac{1}{q(n)}$ таких $x$, что $max S_i(x) < \frac{1}{poly(n)}$, мало.
    \end{idea}
\end{proof}

\end{document}
