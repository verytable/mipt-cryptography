\documentclass[a4paper]{article}

\usepackage{cmap}                   % поиск в PDF
\usepackage[T2A]{fontenc}           % кодировка
\usepackage[english,russian]{babel} % локализация и переносы
\usepackage[utf8x]{inputenc}
\usepackage{amsmath,amsthm}
\usepackage{multirow}

\title{Криптография, Лекция № 13}

\theoremstyle{definition}
\newtheorem{definition}{Definition}

\theoremstyle{plain}
\newtheorem{theorem}{Theorem}
\newtheorem{lemma}{Lemma}
\newtheorem{remark}{Remark}
\newtheorem{claim}{Claim}
\newtheorem{idea}{Idea}     
\newtheorem{example}{Example} 
      
\begin{document}
\maketitle

\noindent Продолжая тему прошлой лекции сегодня поговорим об общем методе превращающем пртокол, надежный в 
получестной модели в протокол, надежный в общей модели. ~\\

\noindent Архитектура состоит из трех фаз:
\begin{enumerate}
	\item Привязка ко входу (что-то такое отправить, после чего нельзя подменить вход)
	\item Генерация общих случайных битов
	\item Симуляция протокола из получестной модели
		(ключевой элемент - доказательства с нулевым разглашением)
\end{enumerate}

\noindent Схемы привязки ($\rightarrow$ подбрасывание монетки по телефону)
$\rightarrow$ ZKP ( $\rightarrow$ передача занчения функции,
когда принимающая сторона знает хеш $(\alpha, h(\alpha)) \mapsto (\varepsilon, f(\alpha))$ ) $\rightarrow$
ZKPoK $\rightarrow$ передача значения функции. Существует еще обобщенное
подбрасывание монетки, авторизованные вычисления, привязка ко входу. Из этого 
всего получается итоговый протокол (здесь должна быть схема).~\\

\noindent ZKP для языка $\{(u, v): \exists \alpha\ h(\alpha) = u, f(\alpha) = v\}$~\\
\noindent Передача значения: $(\alpha, 1^{|\alpha|}) \mapsto (\varepsilon, f(\alpha))$.~\\
\noindent Авторизованные вычисления:
$(\alpha, \beta) \mapsto if (\beta == h(\alpha))\quad (\varepsilon, f(\alpha))\quad else\quad (\varepsilon, (h(\alpha), f(\alpha)))$~\\

\subsection{Протокол}

\begin{enumerate}
	\item Первая сторона выбирает случайное $r' \in \{0, 1\}^n, s \in \{0, 1\}^{n^2}$
	\item При помощи протокола передачи образа передается привязка $c_s(r')$
	\item Обычным образом выбирается общее случайное $r''$.
	\item Итоговое $r = r' \oplus r''$
	\item Использутеся протокол авторизованных вычислений для $\alpha = (s, r', r'')$,
		$\beta = h(\alpha) = (c_s(r'), r'')$, $f(\alpha) = g(r' \oplus r'')$ 
\end{enumerate}

\noindent Если все по честному, то протокол работает правильно. Что может плохого сделать
первая сторона? Может выбирать $r'$ не случайно, но ей все-равно придется привязаться, и
она не сможет потом его поменять, тогда $r''$ будет по честному случайным и, следовательно, 
$r$ будет по честному случайным.~\\

\noindent Привязка ко входу: $(x, 1^{|x|}) \mapsto (r, c_r(x))$, где $r$ случайное из $\{0, 1\}^n$.~\\

\noindent Считаем, что $n = |x|$.

\begin{enumerate}
	\item $S$ выбирает случайное $r' \in \{0, 1\}^{n^2}$
	\item Протокол передачи значения: посылает привязку к $x$ - $c_{r'}(x)$
	\item При помощи обобщенного подбрасывания монетки $S$ получает $(r, r'')$, а $R$
		получает $c_{r''}(r)$. Здесь $r \in \{0, 1\}^{n^2}$, $r'' \in \{0, 1\}^{n^3}$
	\item Авторизованные вычисления, где $\alpha = (x, r, r', r'')$,
		$\beta = h(\alpha) = (c_{r'}(x), c_{r''}(x))$, $f(\alpha) = c_r(x)$
\end{enumerate}

\noindent Такая многоступенчатая схема нужна для невозможности подменить $x$ и $r$.~\\

\end{document}
