\documentclass{report}%
\usepackage{amsmath}%
\usepackage{amsfonts}%
\usepackage{amssymb}%
\usepackage{graphicx}
\usepackage[T2A]{fontenc}
\usepackage[utf8x]{inputenc}
\usepackage[english, russian]{babel}
\usepackage{csquotes}
%----------------------------------------------------------
\newtheorem{theorem}{Теорема}
\newtheorem{problem}{Задача}
\newtheorem{claim}{Утверждение}
\newtheorem{definition}{Определение}
\newtheorem{lemma}{Лемма}
\newtheorem{remark}{Замечание}
\newtheorem{question}{Вопрос}
\newtheorem{example}{Пример}
\newtheorem{exercise}{Упражнение}
\newenvironment{answer}{\par\noindent{\bf Ответ.}}{\hfill$\scriptstyle\blacksquare$}
\newenvironment{proof}{\par\noindent{\bf Доказательство.}}{\hfill$\scriptstyle\blacksquare$}
\newenvironment{solution}{\par\noindent{\bf Решение}}{\hfill$\scriptstyle\blacksquare$}
%----------------------------------------------------------
\begin{document}
\frenchspacing
%\numberwithin{equation}{chapter}
\binoppenalty=10000
\relpenalty=10000
\setcounter{secnumdepth}{-1}
\title{Индивидуальный проект~\\Геометрическая криптография}
\author{Арсений Балобанов}
\date{\today}
\maketitle


\section{Введение}

Геометрическая криптография - отдельная область криптографии, впервые предложенная к изучению
Ади Шамиром, Рональдом Линн Ривестом и Майком Бурместером в 1996 году \cite{trisect}.
В геометрической криптографии в качестве сообщений и шифров выступают
такие геометрические объекты как угол и интервал, а
вычисления производятся с помощью циркуля и линейки. В основу протоколов геометрической
криптографии ложатся неразрешимые или трудноразрешимые задачи геометрии, например, удвоение куба,
квадратура круга. Методы геометрической криптографии мало применимы на практике, однако, они
расширяют аудиторию науки в целом и могут быть использованы в педагогике
в качестве поясняющих примеров более сложных криптографических протоколов.

\section{Геометрическая криптография}

Современная теория алгоритмов основывается на представлении данных
в виде последовательностей символов (обычно битов), и осуществлению над ними операций
из довольно небольшого набора (например, конъюнкция и дизъюнкция).
Но вычисления можно производить и над другими представлениями данных. Например, довольно хорошо
изучена модель построения циркулем и линейкой. Напомним стандартные операции, разрешенные в
построении:
\begin{enumerate}
	\item Через две различные точки можно провести единственную прямую
	\item Можно построить точку пересечения двух различных пересекающихся прямых
	\item Имея две различные точки $A$ и $B$ можно построить три точки $C_1, C_2, C_3$
		отличные от $A$ и $B$ такие что:
		\begin{enumerate}
			\item $C_1$ лежит на прямой проходящей через $A$ и $B$ между $A$ и $B$
			\item $C_2$ лежит на прямой проходящей через $A$ и $B$, и $B$ между $A$ и $C_2$
			\item $C_3$ не лежит на прямой проходящей через $A$ и $B$
		\end{enumerate}
	\item Для интервала $AB$ и луча $CD$ можно построить точку $E$ на луче $CD$ такую, что
		$AB$ и $CE$ конгруэнтны
	\item Можно построить точку (точки) пересечения окружности и пересекающей ее прямой
\end{enumerate}

\noindent Подробнее об аксиомах и аксиоматических понятиях,
таких как ``точка'', ``прямая'', ``конгуэнтность'' можно прочитать в \cite{britt}.~\\

\noindent Неотъемлемой частью криптографических протоколов является возможность 
создания ``секретов'', скрытых от противника. Они не могут быть построены циркулем и линейкой
с помощью имеющихся объектов, иначе, противник сможет найти их. Поэтому вводится дополнительная
аксиома выбора случайных точек:
\begin{enumerate}
  \setcounter{enumi}{5}
  \item На единичной окружности можно построить случайную точку
\end{enumerate}

\section{Протокол идентификации}

Алиса (Прувер) хочет установить способ доказательства описания своей личности Бобу (Верификатору).

\subsection{Инициализация}

Алиса строит случайный угол $X_A$ с помощью аксиомы 6. Затем строит угол $Y_A$ равный
утроенному углу $X_A$, после построения публикует копию угла $Y_A$. Задача утроения угла
подвластна любому школьнику знакомому с геометрией, а вот задача деления угла на три равных, наоборот,
не разрешима. Трисекция угла выступает своего рода односторонней функцией в геометрической
криптографии, поэтому Алиса уверена, что только она знает величину угла $X_A$.

\subsection{Протокол}

\begin{enumerate}
	\item Алиса передает Бобу копию угла $R$, который она построила утроением угла $K$, который она
		построила случайным образом
	\item Боб подкидывает монетку и сообщает Алисе результат
	\item Если Боб говорит ``орел'', Алиса передает Бобу копию угла $K$ и Боб проверяет,
		что $3 \cdot K = R$~\\
		Если Боб говорит ``решка'', Алиса передает Бобу копию угла $L = K + X_A$, и Боб
		проверяет, что $3 \cdot L = R + Y_A$
\end{enumerate}

\noindent Эти три шага повторяются независимо $t$ раз, и Боб принимает доказательство Алисы
только если все $t$ проверок успешны.

\begin{claim}~\\
	Описанный выше протокол является протоколом доказательства знания угла $X_A$ (личности Алисы),
	с ошибкой $2^{-t}$, а также протоколом с нулевым разглашением.
\end{claim}

\begin{proof}~\\
	Если Алиса и Боб будут следовать протоколу, то, ясно что, Боб примет доказательство Алисы.~\\
	Самозванец же, который не знает угол $X_A$, не сможет построить оба угла $L$ и $K$, иначе
	он смог бы построить угол $L - K = X_A$. Поэтому, Боб примет доказательство Алисы с вероятностью
	не более чем $\frac{1}{2}$ на каждой итерации, и следовательно, с вероятностью не более
	$2^{-t}$ на всех $t$ итерациях. Отсюда следует, что данный протокол - протокол доказательства
	знания (Proof of knowledge) $X_A$ \cite{interact}.~\\
	Для доказательства нулевого разглашения, будем симулировать то, что ``видит'' Боб во
	время исполнения протокола. Боб ``видит'' сообщения Алисы и свои броски монеты. Эти состояния
	можно записать как тройки ($R$,~``орел'',~$K$) и ($R$, ``решка'', $L$). Для симуляции владельца
	секрета, можно выбрать $K$ случайно, и взять $R = 3 \cdot K$, выбрать случайное $L$ и разрешить
	$3 \cdot L = R + Y_A$ относительно $R$. Таким образом, Боб может симулировать действия Алисы,
	тем самым получает нулевое разглашения относительно $X_A$.
\end{proof}

\begin{theorem}(Гаусса - Ванцеля)~\\
	Правильный $n$-угольник можно построить с помощью циркуля и линейки тогда и только тогда, когда
	$n = 2^k \cdot p_1 \cdot \ldots \cdot p_m$, где $p_i$ - различные простые числа Ферма ($2^{2^a} + 1$).
\end{theorem}

\begin{theorem}(Ванцель)~\\
	Задача трисекции угла $\alpha$ разрешима тогда и только тогда, когда
	разрешимо в квадратных радикалах уравнение
	$$
		4x^3  - 3x  - cos(\alpha) = 0
	$$
\end{theorem}

\section{Применения протокола идентификации}

Геометрическая криптография может адаптировать многие криптографические примитивы.
Рассмотрим два расширения описанного протокола: параллельное исполнение и ``множественный секрет''.~\\

\noindent Для параллельного исполнения $t$ итераций протокола выполняются одновременно:
\begin{itemize}
	\item Алиса посылает Бобу $t$ копий углов $R_i$
	\item Боб подбрасывает $t$ монет
	\item Алиса посылает Бобу копии соответствующих углов (шаг 3)
\end{itemize}

\noindent Ожидаемое число испытаний в этом случае - $2^{t}$.
Этот протокол может быть реализован, так как не накладывается никаких сложностных ограничений на построения.

\begin{remark}~\\
	Протокол параллельной идентификации также будет являться протоколом с
	нулевым разглашением, в отличие от традиционной криптографии, в которой
	не известно будет ли параллельное исполнения протокола с нулевым разглашением
	протоколом с нулевым разглашением.
\end{remark}

\noindent Для множественного секрета Алиса строит $k$ случайных углов $X_{A_1}, \ldots, X_{A_k}$
и публикует их утроения $Y_{A_1}, \ldots, Y_{A_k}$. Затем Алиса доказывает Бобу, что она
знает все углы $X_{A_1}, \ldots, X_{A_k}$ с помощью следующего протокола, повторенного $t$ раз:

\begin{enumerate}
	\item Алиса посылает Бобу копию угла $R$, построенного утроением случайного угла $K$
	\item Боб посылает Алисе строку битов $b_1, \ldots, b_k$ - результат подбрасывания $k$ монет
	\item Алиса посылает Бобу копию угла $L = K + \sum_{i = 1}^{i = k} b_i X_{A_i}$
		и Боб проверяет, что $3 \cdot L = R + \sum_{i = 1}^{i = k} b_i Y_{A_i}$
\end{enumerate}

\noindent Боб принимает доказательство Алисы только в случае принятия на всех $t$ итерациях.
Легко видеть, что ошибка для этого протокола уже $2^{-kt}$.

\subsection{Протокол аутентификации}

Пусть $m$ - целое число, которое Алиса хочет подтвердить. Это ограничение не сильное
в виду того, что Алиса может опирать только на геометрические объекты, которые можно
построить циркулем и линейкой, коих счетно.~\\

\noindent В протоколе Алиса строит два угла $X_{A_1}$ и $X_{A_2}$, и публикует их утроения
$Y_{A_1}$ и $Y_{A_2}$. Алиса доказывает Бобу, что она знает угол $Z = m \cdot X_{A_1} + X_{A_2}$
использую протокол идентификации, рассмотренный выше:

\begin{enumerate}
	\item Алиса посылает Бобу копию угла $R$, построенного утроением случайного угла $K$
	\item Боб подбрасывает монетку, и сообщает Алисе бит $b$
	\item Алиса посылает Бобу угол $L = K + b(m \cdot X_{A_1} + X_{A_2})$~\\
		Боб проверяет, что $3 \cdot L = R + b(m \cdot Y_{A_1} + Y_{A_2})$
\end{enumerate}

\subsection{Обобщение протокола идентификации}

Пусть $N$ - дополнительный угол, используемый как модуль при выполнении построений
суммы и разности углов. Пусть $Y$ - угол, который нужно разделить на три по модулю $N$.
Тогда у уравнения $Y = 3 \cdot X\ mod(N)$ имеется три решения, которые отличаются на
множители $\frac{N}{3}$. Поэтому каждый, кто знает два решения $X_1$ и $X_2$ может
осуществлять трисекцию по модулю $N$. Подобное свойство позволяет обобщать многие 
криптографические конструкции, использующие теорию чисел, в которых знание двух
различных корней числа $y$ по модулю $n$ позволяет факторизовать $n$. Рассмотрим
одно применение.~\\

\noindent Предположим, что распределение точек в аксиоме 6 не равномерное,
тогда описанный протокол идентификации уже не будет протоколом с нулевым разглашением,
потому что уже нельзя будет симулировать ($R$, ``решка'', $L$) выбирая случайное $L$ и решая
$3 \cdot L = R + Y_A$ (так как $R$ может иметь другое распределение). Поэтому Боб может
получить некоторую информацию и построить ``секретный'' угол Алисы -  $X_A$. Покажем,
как использовать соображения выше для усиления безопасности протокола в этом случае.~\\
Во-первых, распределение углов должно быть непрерывным, в частности, вероятность выбора
конкретного угла $K$ должна быть ноль. Иначе Боб сможет построить углы $K$ и $L = K + X_A$, а
из них построить $X_A$. Пусть $N$ - угол, который не может разделить на три части ни Алиса, ни Боб
(например, $N$ выбирается случайно доверенным лицом).
Для защиты Алиса будет сообщать Бобу не абсолютные значения углов, а их значения по модулю $N$.
Подобный шаг не сделает распределение углов равномерным, но сделает абсолютно не понятным для
Боба какая именно из трех трисекций угла $Y_A$: $X_A$, $X_A + \frac{N}{3}$, $X_A + 2 \frac{N}{3}$
используется Алисой в $L$. Даже если Боб будет иметь три такие возможности, он выберет правильную
с вероятностью не более $\frac{1}{3}$. Если предположить, что эту схему можно взломать и Боб может
построить трисекцию $Y_A$, с вероятностью $\frac{2}{3}$ это будет не та трисекция, использованная Алисой,
поэтому подобное построение с некоторой фиксированной вероятностью приведет к построению трисекции угла $N$.

\begin{thebibliography}{1}

  	\bibitem{trisect} Mike Burmester, Ronald L Rivest and Adi Shamir
  		{\em ``Geometric Cryptography Identification by Angle Trisection''}\quad
  		e-Prints, US Department of Energy, OSTI. Retrieved 19 June 2014.

	\bibitem{britt} {\em The New Encyclopedia Brittanica}, Volume 19, Geometry, pages 887--936, 1995.
	
	\bibitem{interact} S. Goldwasser, S. Micali and C. Rackoff.
		{\em ``The Knowledge Complexity of Interactive Proof Systems''}\quad
		Siam J. Comput., Vol 18 (1), pages 186--208, 1989.	
	
\end{thebibliography}

\end{document}

\end{document}
