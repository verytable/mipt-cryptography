\documentclass[a4paper]{article}

\usepackage{cmap}                   % поиск в PDF
\usepackage[T2A]{fontenc}           % кодировка
\usepackage[english,russian]{babel} % локализация и переносы
\usepackage[utf8x]{inputenc}
\usepackage{amsmath,amsthm}

\title{Криптография, Лекция № 2}

\theoremstyle{definition}
\newtheorem{definition}{Definition}

\theoremstyle{plain}
\newtheorem{theorem}{Theorem}
\newtheorem{lemma}{Lemma}
\newtheorem{remark}{Remark}
\newtheorem{claim}{Claim}
\newtheorem{idea}{Idea}      
      
\begin{document}
\maketitle

\noindent Вспомним определения сильно односторонних функций:
$$
	\forall \{R_n\}\ \exists p(\cdot)\ \exists N\ \forall n > N\ Pr\{f(R_n(f(x))) = f(x)\} < 1 - \frac{1}{p(n)}
$$ 
$$
	\exists p(\cdot)\ \forall \{R_n\}\ \exists N\ \forall n > N\ Pr\{f(R_n(f(x))) = f(x)\} < 1 - \frac{1}{p(n)}
$$ 

\noindent Из второго всегда следует первое.

\noindent Отрицание второго:
$$
	\forall p(\cdot)\ \exists \{R_n\}\ \forall N\ \exists n > N\ Pr\{f(R_n(f(x))) = f(x)\} \ge 1 - \frac{1}{p(n)}
$$ 

\noindent Можно найти такую последовательность из отрицания выше:
$$
	\forall k\ \exists \{R_n^{(k)}\}\ \forall N\ \exists n > N\ Pr\{f(R_n(f(x))) = f(x)\} \ge 1 - \frac{1}{n^k}
$$

$$
	R_{n_1^{(1)}} \colon Pr\{\cdots\} \ge 1 - \frac{1}{n_1}
$$
$$
	R_{n_2^{(2)}} \colon n_2 > n_1, Pr\{\cdots\} \ge 1 - \frac{1}{n_2^2}
$$
\noindent И так далее.

\noindent Из этого ряда вытекает следующее:
$$
	\exists \{R_{n}\}\ \forall k\ \forall N\ \exists n = max\{n_k, n_k > N\} > N Pr\{\cdots\} \ge 1 - \frac{1}{n^k}
$$

$$
	\exists p(\cdot)\ \forall \{R_n\}\ \exists N\ \forall n > N\ Pr\{f(R_n(f(x))) = f(x)\} < 1 - \frac{1}{p(n)}
$$

$$
	F(x_1, \ldots, x_N) = f(x_1)\ldots f(x_N)
$$

\begin{theorem}~\\
	Если $N = n \cdot p(n)$, то $F$ - сильно односторонняя, то есть
	$$
		\forall q(\cdot)\ \forall\{T_n\}\ \exists N\ \forall n > N\ Pr\{F(T_n(F(x))) = F(x)\} < \frac{1}{q(n)}
	$$
\end{theorem}

\begin{proof}~\\
	Пусть 
	$$
		\exists q(\cdot)\ \exists \{T_n\}\ \forall N\ \exists n > N\ Pr\{F(T_n(F(x))) = F(x)\} \ge \frac{1}{q(n)}
	$$
	Построим $R_n$:
	$M$ раз повторить процедуру:
	\begin{itemize}
		\item При всех $i$ от $1$ до $N$ выбрать случайные $x_1,\ldots, x_N$
		\item Посчитать $T_n(f(x_1)\ldots f(x_{i - 1})y f(x_{i + 1})\ldots f(x_N))$
		\item Проверить успешность обращения $y$.
	\end{itemize}
	
	\noindent $S_i(x)$ - вероятность успешного обращения $f(x)$ на шаге $i$ внутреннего цикла.~\\
	\noindent $S(x) = max\{S_1(x), \ldots, S_N(x) \}$ - нижняя оценка вероятности обращения на 1 шаге внешнего цикла.
	Все $x$ разделим на две части по предикату $S(x) \ge \delta = \frac{1}{poly(n)}$.~\\
	
	\noindent Для первых, удовлетворяющих предикату, повторение $M = n \cdot \frac{1}{\delta}$ раз приведет к вероятности обращения $f(x)$ $\sim 1 - e^{-n}$~\\
	\noindent Вероятность того, что обращение не сработает ни на одном шаге: $(1 - \delta)^M = (1 - \delta)^{n\cdot \frac{1}{\delta}} \sim e^{-n}$~\\
	
	\noindent Для вторых, трудных, докажем, что таких $x$ мало (например, $< \frac{1}{2p(n)}$)
	
	\noindent Вход $T_n \colon x_1\ldots x_N$ все $x_i$ легкие с вероятностью $(1 - \varepsilon)^N$, где $\varepsilon$ - доля трудных $x$.~\\
	
	\noindent $x_i$ трудный, остальные случайные. В этом случае условная вероятность обращения $< \delta$.~\\
	
	
	\noindent Общая вероятность обращения $\le N\delta\varepsilon + (1 + \varepsilon)^N$ - что-то типа формулы включений-исключений.
	Получим противоречие, если покажем, например, что $N\delta\varepsilon + (1 + \varepsilon)^N < \frac{1}{q(n)}$~\\
	
	\noindent Если $\varepsilon < \frac{1}{2p(n)}$, получим противоречие со слабой одностороностью $F$.~\\
	
	\noindent Если же $\varepsilon \ge \frac{1}{2p(n)}$, возьмем $\delta = \frac{1}{2np(n)q(n)}$. Получим противоречие с предположением, что $F$ не является сильно односторонней.
	
\end{proof}

\noindent Часто в проложениях бывает так, что нельзя сделать так, чтобы функция была везде определена.
	(Например, функция, определенная только на произведении двух больших чисел.) Часто бывает нужно выбрать
	случайный элемент из области определения.

\begin{definition}~\\
	Пусть $\mu_n$ - случайная величина со значениями в $\{0, 1\}^{k(n)}$. Будем говорить, что
	$\mu_n$ - полиномиально вычислимая, если существует полиномиальный вероятностный алгоритм с пустым
	входом $S$ такой, что $\forall x \in \{0, 1\}^{k(n)}\ Pr\{S(\varepsilon) = x\} = Pr\{\mu_n = x\}$
\end{definition}

\begin{definition}~\\
	Статистическое расстояние: $dist(\mu_n, \nu_n) = max_{T \subset \{0, 1\}^{k(n)}} |Pr\{\mu_n \ in T \} - Pr\{\nu_n \in T \}| = \sum_{x \ in \{0, 1\}^{k(n)}} \frac{1}{2}|Pr\{\mu_n = x \} - Pr\{\nu_n = x \}|$
\end{definition}

\begin{definition}~\\
	$\mu_n$ доступна, если
	$$
		\exists \nu_n\ \forall p(\cdot)\ \exists N\ \forall n > N dist(\mu_n, \nu_n) < \frac{1}{p(n)}
	$$
	$\nu_n$ - полиномиально вычислима.
\end{definition}

\begin{definition}~\\
	$f\colon D \rightarrow \{0, 1\}^n$ - частичная односторонняя функция есил:
	\begin{itemize}
		\item Равномерная случайная величина на $D$ доступна
		\item $f$ вычислима за полиномиальное время
		\item $\forall R_n\ \forall p(\cdot)\ \exists N\ \forall n > N\ Pr_{x \in D}\{R_n(f(x)) \in D, f(R_n(f(x))) = f(x)\} < \frac{1}{p(n)}$
	\end{itemize}
\end{definition}

\noindent Предположительные частично односторонние функции:
\begin{itemize}
		\item Функция Рабина: $(x, y) \rightarrow (x^2\ mod\ y, y)$~\\
			$y = p \cdot q$~\\
			$p_n, q$ - простые вида $4k + 3$~\\
			$x$ - квадратичный вычет.
		\item Функция RSA: $(x, y, z) \rightarrow (x^z\ mod\ y, y, z)$ - извлечение корня степени $z$~\\
			$y = p \cdot q$~\\
			$x$ - вычет, взаимно простый с $y$~\\
			$z$ - взаимно просто с $\phi(p\cdot q) = (p - 1)(q - 1)$
		\item Дискретная экспонента $(x, y, z) \rightarrow (x, y, x^z)$ - логарифм по основанию~$x$.
	\end{itemize}

\end{document}
