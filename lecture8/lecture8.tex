\documentclass[a4paper]{article}

\usepackage{cmap}                   % поиск в PDF
\usepackage[T2A]{fontenc}           % кодировка
\usepackage[english,russian]{babel} % локализация и переносы
\usepackage[utf8x]{inputenc}
\usepackage{amsmath,amsthm}
\usepackage{multirow}

\title{Криптография, Лекция № 8}

\theoremstyle{definition}
\newtheorem{definition}{Definition}

\theoremstyle{plain}
\newtheorem{theorem}{Theorem}
\newtheorem{lemma}{Lemma}
\newtheorem{remark}{Remark}
\newtheorem{claim}{Claim}
\newtheorem{idea}{Idea}     
\newtheorem{example}{Example} 
      
\begin{document}
\maketitle

\section{Протоколы идентификации}

Примеры: залогиниться на сайте; доказать в банкомате, что вы владелец. Общая задача:
тот кто должен может зарегестрироваться, а тот кто не должен не мог.
Бывают идентификации на сайте, обычно нужно написать логин и пароль.
Хранить пароль открытым текстом на сервере - понятно, что плохо.
Можно хранить в закрытом виде, и сравнивать хэш значения (то есть вызывать конкретную программу,
иначе злоумышленник может перехватить  само хэш значение; а лучше еще и уметь доказывать, что
ты пользуешься именно этой программой). Другой тип атак - фальшивые банкоматы, жертва вставляет 
карточку, вводит пин, ничего не происходит и жертва думает, что просто банкомат сломался.
Подобное есть и на сайтах - phishing. Цель построить протокол, который спасет и от
поддельных серверов.

\subsection{Протокол с закрытым ключом}

Здесь будет две стороны $P$ - proover, и $V$ - verificator. Через закрытый канал
они оба знают $d$ - закрытый ключ. Соответсвенно, есть протокол (вероятностный алгоритм)
генерации ключей $K$.
 $P$, $V$ - интерактивные вероятностные алгоритмы. В конце $V$ выдает $0$ или $1$.~\\
 
\noindent Условия:
\begin{enumerate}
	\item Корректность:
		$$
			Pr\{V^{P(d)}(d) = 1\} \simeq 1	
		$$
	\item Надежность: Злоумышленник запустил программу $C$ она $k$ раз пообщалась с правильным $P$
		после этого она пошла к верификатору и попыталась выступить в качестве прувера. Сделать это
		ей не должно удасться.
		$$
			\forall C\ \forall k\ Pr\{V^{C^{P^k(d)}}(d) = 1\} \simeq 0		
		$$
\end{enumerate}

\noindent Потребуются псевдослучайные функции. Что это такое? Что значит делать запросы
к случайной функции? Это значит, что если зафиксированы области значения и определения, то случайно
выбирается одна из понятно скольких (двойная экспонента). Псевдослучайных функций хорошо бы
чтобы была одинарная экспонента, чтобы можно было записывать номер строкой полиномиальной длины.


\begin{definition}~\\
	ПСФ - семейство $f_e\colon \{0, 1\}^n \mapsto \{0, 1\}^n$, $e \in \{0, 1\}^{k(n)}$.
	\begin{itemize}
		\item Вычислимость: $e, x \mapsto f_e(x)$ - вычисляется полиномиальным алгоритмом.
		\item Неотличимость: пусть $C_n$ - схемы полиномиального размера
			с оракульным доступом к $f$ (то есть алгоритмы с 
			подсказкой для всех входов данной длины и оракулом)
			$$
				\forall C_n\ Pr_e\{C_n^{f_e} = 1\} \simeq Pr_f\{C_n^f = 1\}
			$$
			Идея в том, что эта схема может вычислить значения в полиномиальном числе точек
			и пытается отличить.
	\end{itemize}
	Конструкция: рассматривается генератор псеводослучайных чисел
	$G\colon \{0, 1\}^n \mapsto \{0, 1\}^{2n}$. Если $e = \varepsilon$, то $f_e(x) = x$.
	$G(x) = G_0(x)G_1(x)$ ($G_0(x), G_1(x)$ - два слова длины $n$).
	Если $|e| = 1$, то $f_e(x) = G_e(x)$. Если $|e| = 2, e = e_1 e_2$,
	то $f_e(x) = G_{e_2}(G_{e_1}(x))$. И так далее.
\end{definition}

\noindent Закрытый ключ - индекс ПСФ. $P$ говорит $V$ вычисли функцию в такой то точке:
$$
	P \xleftarrow{x} V
$$
$$
	P \xrightarrow{f_d(x)} V
$$

\subsection{Протокол с открытым ключом}

Здесь $P$ знает $d$ закрытый ключ, а $V$ - знает $e$ - открытый ключ, который известен вообще всем.~\\

\noindent Условия:
\begin{enumerate}
	\item Корректность:
		$$
			Pr\{V^{P(d)}(e) = 1\} \simeq 1	
		$$
	\item Надежность: 
		$$
			\forall C\ \forall k\ Pr\{V^{C^{P^k(d)}(e)}(e) = 1\} \simeq 0		
		$$
\end{enumerate}

\noindent Построим протокол на базе функции Рабина: есть некоторый модуль $m = p \cdot q$, где
$q, p$ - простые числа из $n$ бит; $(x, m) \mapsto (x^2\ mod\ m, m)$ ($(x, m)$ - закрытый ключ,
а $(x^2 mod m, m)$ - открытый).
$$
	P \xrightarrow{z = y^2\ mod\ m} V
$$
$$
	P \xleftarrow{\alpha \in \{0, 1\}} V
$$
$$
	P \xrightarrow{u = y\cdot x^{\alpha}} V
$$
И, наверное, это нужно повторить много раз, ибо $P$ может заранее знать $\alpha$ и подготовиться.
Вконце $V$ проверяет, что $u^2 = z\cdot (x^2)^{\alpha}\ (mod\ m)$. Если взломщик знает
$u_0 = y \cdot x^{0} = y$ и $u_1 = y \cdot x^{1} = y \cdot x$, то он узнаёт $x = \frac{u_1}{u_0}$.
Почти навероное при разных запусках будут разные $y$. Если $\alpha = 1$, то он узнает квадрат,
но если функция рабина не обратима, то это ему не поможет.~\\

\noindent Итуитивно понятно, что надежность эквивалентна двум вещам:
\begin{enumerate}
	\item
		$$
			Pr\{V^{C(e)}(e) = 1\} \simeq 0		
		$$
	\item У алгоритма $P$ нулевое разглашение.
\end{enumerate}

\noindent Покажем первое (через необратимость (односторонность) функции рабина):
пусть $C$ обманывает $V$: $C$ выбирает $z$, после этого запускает её для $\alpha = 0$ и для $\alpha = 1$.
$C$ выдает $u_0$ и $u_1$ и тогда $x = \frac{u_1}{u_0}$. Это общая идея, которую нужно далее уточнять.~\\

\noindent Про второе: кажется, что в этом случае будет даже статистически нулевое разглашение.

\end{document}